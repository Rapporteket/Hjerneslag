\documentclass [norsk,a4paper,twoside]{article}\usepackage[]{graphicx}\usepackage[]{color}
%% maxwidth is the original width if it is less than linewidth
%% otherwise use linewidth (to make sure the graphics do not exceed the margin)
\makeatletter
\def\maxwidth{ %
  \ifdim\Gin@nat@width>\linewidth
    \linewidth
  \else
    \Gin@nat@width
  \fi
}
\makeatother

\definecolor{fgcolor}{rgb}{0.345, 0.345, 0.345}
\newcommand{\hlnum}[1]{\textcolor[rgb]{0.686,0.059,0.569}{#1}}%
\newcommand{\hlstr}[1]{\textcolor[rgb]{0.192,0.494,0.8}{#1}}%
\newcommand{\hlcom}[1]{\textcolor[rgb]{0.678,0.584,0.686}{\textit{#1}}}%
\newcommand{\hlopt}[1]{\textcolor[rgb]{0,0,0}{#1}}%
\newcommand{\hlstd}[1]{\textcolor[rgb]{0.345,0.345,0.345}{#1}}%
\newcommand{\hlkwa}[1]{\textcolor[rgb]{0.161,0.373,0.58}{\textbf{#1}}}%
\newcommand{\hlkwb}[1]{\textcolor[rgb]{0.69,0.353,0.396}{#1}}%
\newcommand{\hlkwc}[1]{\textcolor[rgb]{0.333,0.667,0.333}{#1}}%
\newcommand{\hlkwd}[1]{\textcolor[rgb]{0.737,0.353,0.396}{\textbf{#1}}}%

\usepackage{framed}
\makeatletter
\newenvironment{kframe}{%
 \def\at@end@of@kframe{}%
 \ifinner\ifhmode%
  \def\at@end@of@kframe{\end{minipage}}%
  \begin{minipage}{\columnwidth}%
 \fi\fi%
 \def\FrameCommand##1{\hskip\@totalleftmargin \hskip-\fboxsep
 \colorbox{shadecolor}{##1}\hskip-\fboxsep
     % There is no \\@totalrightmargin, so:
     \hskip-\linewidth \hskip-\@totalleftmargin \hskip\columnwidth}%
 \MakeFramed {\advance\hsize-\width
   \@totalleftmargin\z@ \linewidth\hsize
   \@setminipage}}%
 {\par\unskip\endMakeFramed%
 \at@end@of@kframe}
\makeatother

\definecolor{shadecolor}{rgb}{.97, .97, .97}
\definecolor{messagecolor}{rgb}{0, 0, 0}
\definecolor{warningcolor}{rgb}{1, 0, 1}
\definecolor{errorcolor}{rgb}{1, 0, 0}
\newenvironment{knitrout}{}{} % an empty environment to be redefined in TeX

\usepackage{alltt}	%, report
\addtolength{\hoffset}{-0.5cm}
\addtolength{\textwidth}{1cm}
\addtolength{\voffset}{-1cm}
\addtolength{\textheight}{2cm}

\usepackage{rotating} %add rotating for plain tables
\usepackage{pdflscape} %add rotating/landcape for pdf
%%for nice looking tabs
\usepackage{booktabs}

\usepackage[norsk]{babel}
\usepackage[utf8x]{inputenc}
\usepackage{textcomp}
\usepackage{fancyhdr}
\pagestyle{fancy}
\usepackage{amsmath}

%add rotating for plain tables
\usepackage{rotating}

%add rotating/landcape for pdf
\usepackage{pdflscape}

%add long tables
\usepackage{longtable}
\usepackage{afterpage} 
\afterpage{\clearpage} %unngå sideskift etter floatflsuh
%\restylefloat{figure} %gjør det mulig å angi H som parameter for plassering av floats

%for nice looking tabs
\usepackage{booktabs}


%bytte font
\renewcommand{\familydefault}{\sfdefault}

%setter grå skrift fremfort sort
\usepackage{xcolor}
\usepackage{graphicx}
\definecolor{SKDE}{rgb}{0,0.32,0.61}
\definecolor{lysblaa}{rgb}{0.27,0.51,0.71}
\definecolor{moerkblaa}{rgb}{0.0,0.0,0.47}
\definecolor{lysgraa}{rgb}{0.8,0.8,0.8}
\definecolor{middelsgraa}{rgb}{0.5,0.5,0.5}
\definecolor{moerkgraa}{rgb}{0.25,0.25,0.25}
\color{moerkgraa}

\usepackage[pdftex, colorlinks, linkcolor=lysblaa, urlcolor=lysblaa]{hyperref}

%bytte overskrifter 
\usepackage[compact]{titlesec}
\titleformat{\section} {\vspace*{10pt}\color{SKDE}\normalfont\Large\bfseries} {\thesection}{}{}
\titleformat{\subsection} {\color{middelsgraa}\normalfont\large\bfseries} {\thesection}{}{}

%topptekst og vertikal sidenummer
\fancyhead{}\fancyfoot{}  % clear all fields
\fancyheadoffset[LO, RE]{3cm}
\fancyfootoffset[LO]{1.5cm}
\fancyfootoffset[RE]{1.5cm}
\fancyhead[LO]{\colorbox{SKDE}{\textcolor{white}{\hspace*{2cm}\scshape\small Norsk Hjerneslagregister}}} %Lengde på stripa
\fancyfoot[LO]{\colorbox{SKDE}{\textcolor{white}{\scshape\small\thepage}}} 
\fancyfoot[RE]{\colorbox{SKDE}{\textcolor{white}{\scshape\small\thepage}}}
\renewcommand{\headrulewidth}{0pt} %\iffloatpage{0pt}{0.4pt}
 \renewcommand{\footrulewidth}{0pt}
%evt. horisontal sidenummer
\fancyfoot[LO]{\colorbox{SKDE}{\textcolor{white}{\hspace*{2cm} \small \thepage}}} \fancyfootoffset[LO]{4.4cm}
\fancyfoot[RE]{\hspace*{2cm}\colorbox{SKDE}{\textcolor{white}{\small \thepage \hspace*{3cm}}}}  \fancyfootoffset[RE]{5.3cm}
\setcounter{secnumdepth}{-1} 


\title{Resultater fra Norsk Hjerneslagregister}
%\author{Hjerneslagregisteret og Lena Ringstad Olsen (SKDE)}
%\fancyhead[R]{\includegraphics[height=2cm]{Hjerneslaglogo.pdf}}
\IfFileExists{upquote.sty}{\usepackage{upquote}}{}
\begin{document}
\maketitle
\tableofcontents
%\newpage
\listoffigures
\listoftables
\newpage


%% this is equivalent to SweaveOpts{...}







\section{Innledning}
Her presenteres en rekke tabeller og figurer som viser resultater fra Norsk Hjerneslagregister.
Rapporten er utarbeidet av Lena Ringstad Olsen ved SKDE, på bestilling fra ledelsen i Norsk Hjerneslagregister.
I denne rapporten vises aggregerte tall for \textbf{St. Olav} og for hele landet. 
Tabellene og figurene er basert
på alle registrerte innleggelser i perioden 2013-01-01 til 2016-02-06. 
Alle figurer, og noen av tabellene, finner man igjen som forhåndsdefinerte figurer/tabeller 
i Rapporteket. Når man er logget inn på Rapporteket, kan man se på figurene med egendefinerte 
utvalg av dataene. F.eks. hvis man
ønsker å se på resultater for kvinner og menn adskilt eller for en bestemt diagnose.
For alle figurer i Rapporteket kan man gjøre følgende utvalg:
\begin{itemize}
	\item Alder: f.o.m.-t.o.m.
	\item Tidsperiode: f.o.m.-t.o.m. basert på innleggelsesdato
	\item Kjønn: mann, kvinne, begge 
	\item Diagnose: Blødning, Infarkt, Udef. (I61, I63, I64)
	\item Innlagt innen 4 timer, eller mer enn 4 timer etter symptomdebut
	\item NIHSS score ved innkomst
\end{itemize}


\clearpage
\section{Kvalitetsindikatorer}

\begin{figure}[ht]
{\centering \includegraphics[width= 1\textwidth]{FigKvalInd.pdf} }
\caption{\label{fig:KvalInd} Utvalgte kvalitetsindikatorer}
\end{figure}



\clearpage
\section{Pasientsammensetning og -karakteristika}

Dette kapitlet inneholder resultater for:
\begin{itemize}
	\item Antall registreringer
	\item Alder, kjønn og sivilstatus
	\item Funksjon og boforhold før hjerneslaget
	\item Risikofaktorer og medikamentell behandling før hjerneslaget 
\end{itemize}


%Alle registreringer
For hele landet er det gjort registrering av innleggelser i perioden 
2013-01-01 til 2016-02-06.
Kvinner og menn utgjorde hhv. 46.1\% og 53.9\% av innleggelsene.


%Eget sykehus
Ved St. Olav er det gjort registrering av innleggelser i perioden 
2013-01-01 til 2016-02-01.
Kvinner og menn utgjorde hhv. 48.6\% og 51.4\% av innleggelsene.


% latex table generated in R 3.2.3 by xtable 1.8-2 package
% Tue Mar 08 13:14:16 2016
\begin{table}[ht]
\centering
\begin{tabular}{lrrrrr}
  \hline
 & 2013 & 2014 & 2015 & 2016 & Totalt: \\ 
  \hline
Eget sykehus & 107 & 98 & 102 & 8 & 315 \\ 
  Alle sykehus & 1417 & 1686 & 1502 & 47 & 4652 \\ 
   \hline
\end{tabular}
\caption{Antall registrerte slagtilfeller per år.} 
\label{tab:AntRegEget}
\end{table}
% latex table generated in R 3.2.3 by xtable 1.8-2 package
% Tue Mar 08 13:14:17 2016
\begin{table}[ht]
\centering
\begin{tabular}{lrrrrr}
  \hline
 & 2013 & 2014 & 2015 & 2016 & Totalt \\ 
  \hline
Eget sykehus & 107 & 98 & 101 & 8 & 314 \\ 
  Alle sykehus & 1406 & 1682 & 1492 & 47 & 4609 \\ 
   \hline
\end{tabular}
\caption{Antall registrerte individer per år.} 
\label{tab:AntPasAarEget}
\end{table}




% latex table generated in R 3.2.3 by xtable 1.8-2 package
% Tue Mar 08 13:14:17 2016
\begin{table}[ht]
\centering
\begin{tabular}{lrrr}
  \hline
 & Akuttskjema & Oppfølgingsskjema & Totalt: \\ 
  \hline
Eget sykehus & 315 & 270 & 85.7 \\ 
  Alle sykehus & 4652 & 2991 & 64.3 \\ 
   \hline
\end{tabular}
\caption{Antall akuttskjema og andel av disse med oppfølgingsskjema, inkl. døde.
		Andelen oppfølging kan være noe lavere enn den reelle, siden slag oppstått
		de siste 3 mnd ikke har rukket å få oppfølging.} 
\label{tab:AndelOppfEget}
\end{table}


Tabellene \ref{tab:Pasientkarakteristika1} og \ref{tab:Pasientkarakteristika2} viser pasientkarakteristika  
for hhv. eget sykehus og hele landet. I beregningene for NIHSS ved innleggelse, er pasienter hvor det 
er krysset av for at NIHSS ikke er utført, tatt bort.

\clearpage

% latex table generated in R 3.2.3 by xtable 1.8-2 package
% Tue Mar 08 13:14:17 2016
\begin{table}[ht]
\centering
\begin{tabular}{lrrrrr}
  \hline
 & Middelverdi & Median & Min. & Maks. & Ant. obs \\ 
  \hline
Alder & 74.6 & 77 & 18 & 100 & 315 \\ 
  ...Alder, menn & 71.9 & 74 & 32 & 96 & 162 \\ 
  ...Alder, kvinner & 77.6 & 81 & 18 & 100 & 153 \\ 
  Total liggetid (døgn) & 7.5 & 5 & 0 & 58 & 315 \\ 
  Timer, symptomdebut - innleggelse & 0.0 & 0 & 0 & 0 & 315 \\ 
  Minutter, innleggelse - trombolyse & 3492.3 & 1950 & 960 & 13500 & 44 \\ 
  NIHSS, innleggelse & 6.2 & 3 & 0 & 40 & 242 \\ 
   \hline
\end{tabular}
\caption{Pasientkarakteristika, St. Olav.} 
\label{tab:Pasientkarakteristika1}
\end{table}
% latex table generated in R 3.2.3 by xtable 1.8-2 package
% Tue Mar 08 13:14:17 2016
\begin{table}[ht]
\centering
\begin{tabular}{lrrrrr}
  \hline
 & Middelverdi & Median & Min. & Maks. & Ant. obs \\ 
  \hline
Alder & 74.7 & 77 & 17 & 104 & 4652 \\ 
  ...Alder, menn & 72.0 & 73 & 19 & 98 & 2509 \\ 
  ...Alder, kvinner & 77.7 & 80 & 17 & 104 & 2143 \\ 
  Total liggetid (døgn) & 7.3 & 5 & -113 & 127 & 4651 \\ 
  Timer, symptomdebut - innleggelse & 0.0 & 0 & 0 & 0 & 4652 \\ 
  Minutter, innleggelse - trombolyse & 4623.0 & 2340 & 240 & 874800 & 632 \\ 
  NIHSS, innleggelse & 6.1 & 4 & 0 & 42 & 3534 \\ 
   \hline
\end{tabular}
\caption{Pasientkarakteristika, hele landet.} 
\label{tab:Pasientkarakteristika2}
\end{table}



\begin{figure}[ht]
{\centering \includegraphics[width= 0.75\textwidth]{FigAlder.pdf} }
\caption{\label{fig:Alder} Aldersfordeling for registrerte slagpasienter}
\end{figure}


% latex table generated in R 3.2.3 by xtable 1.8-2 package
% Tue Mar 08 13:14:17 2016
\begin{table}[ht]
\centering
\begin{tabular}{lrrr}
  \hline
 & Kvinner (\%) & Menn (\%) & Ant. reg. \\ 
  \hline
Mod. Rankin Scale $<$=2 & 68.0 & 79.0 & 315 \\ 
  Boligforhold: &  &  & 315 \\ 
  ...Bor hjemme u/hjelp & 56.9 & 80.2 &  \\ 
  ...Bor hjemme m/hjelp & 28.8 & 10.5 &  \\ 
  ...Bor i omsorgsbolig & 6.5 & 3.1 &  \\ 
  ...Bor på sykehjem & 7.8 & 6.2 &  \\ 
  Bor alene & 47.7 & 27.2 &  \\ 
   \hline
\end{tabular}
\caption{Funksjonsnivå og boligforhold før hjerneslaget, St. Olav.} 
\label{tab:Boligforh1}
\end{table}
% latex table generated in R 3.2.3 by xtable 1.8-2 package
% Tue Mar 08 13:14:17 2016
\begin{table}[ht]
\centering
\begin{tabular}{lrrr}
  \hline
 & Kvinner (\%) & Menn (\%) & Ant. reg. \\ 
  \hline
Mod. Rankin Scale $<$=2 & 81.8 & 89.6 & 4647 \\ 
  Boligforhold: &  &  & 4620 \\ 
  ...Bor hjemme u/hjelp & 60.2 & 81.3 &  \\ 
  ...Bor hjemme m/hjelp & 25.2 & 12.9 &  \\ 
  ...Bor i omsorgsbolig & 6.0 & 2.3 &  \\ 
  ...Bor på sykehjem & 8.5 & 3.6 &  \\ 
  Bor alene & 50.6 & 29.1 &  \\ 
   \hline
\end{tabular}
\caption{Funksjonsnivå og boligforhold før hjerneslaget, hele landet.} 
\label{tab:Boligforh2}
\end{table}



\begin{figure}[ht]
{\centering \includegraphics[width= 0.75\textwidth]{FigSivStatusPre.pdf} }
\caption{\label{fig:SivilstatusPre} Sivilstatus da slaget inntraff.}
\end{figure}



For variable
hvor ett av svaralternativene er ''Ukjent'', er dette tatt ut av totalantallet til den enkelte variabel. 
Antall observasjoner er derfor registreringer uten ukjente. Det betyr at vi
antar at fordelinga av variabelalternativene er tilsvarende hos de som har ukjent utfall som de som har
kjent utfall. Det kan være grunn til å anta at de fleste ukjente for variabelen ''Røyking'' er ikke-røykere.
Hvis dette er tilfelle, vil den reelle andelen røykere være noe lavere enn det tabell \ref{tab:Risiko1}
og \ref{tab:Risiko2} viser. Dette gjelder ikke medikamentvariablene. Der beregnes andel som har fått
de ulike medikamentene ut fra totalantallet, dvs. inklusive ukjente.

% latex table generated in R 3.2.3 by xtable 1.8-2 package
% Tue Mar 08 13:14:17 2016
\begin{table}[ht]
\centering
\begin{tabular}{lrr}
  \hline
 & Ja (\%) & Ukjent (\%) \\ 
  \hline
Våken v/innleggelsen & 79.4 & 0.3 \\ 
  Facialisparese & 45.4 & 4.4 \\ 
  Armparese & 63.2 & 2.2 \\ 
  Språk- el taleproblemer & 52.1 & 3.3 \\ 
  Beinparese & 54.0 & 2.9 \\ 
  Minst ett FAST-symptom & 81.6 & 3.2 \\ 
  Andre fokale slagsymptomer & 55.2 & 8.3 \\ 
   \hline
\end{tabular}
\caption{Status i akuttfasen. Andeler av alle registreringer, St. Olav (N=315).} 
\label{tab:StatusAkutt1}
\end{table}
% latex table generated in R 3.2.3 by xtable 1.8-2 package
% Tue Mar 08 13:14:17 2016
\begin{table}[ht]
\centering
\begin{tabular}{lrr}
  \hline
 & Ja (\%) & Ukjent (\%) \\ 
  \hline
Våken v/innleggelsen & 84.3 & 0.6 \\ 
  Facialisparese & 42.4 & 4.1 \\ 
  Armparese & 51.3 & 3.6 \\ 
  Språk- el taleproblemer & 51.0 & 4.4 \\ 
  Beinparese & 44.7 & 4.0 \\ 
  Minst ett FAST-symptom & 78.5 & 3.2 \\ 
  Andre fokale slagsymptomer & 48.7 & 9.5 \\ 
   \hline
\end{tabular}
\caption{Status i akuttfasen. Andeler av alle registreringer, hele landet (N=4652).} 
\label{tab:StatusAkutt2}
\end{table}



% latex table generated in R 3.2.3 by xtable 1.8-2 package
% Tue Mar 08 13:14:17 2016
\begin{table}[ht]
\centering
\begin{tabular}{lrrr}
  \hline
 & Kvinner (\%) & Menn (\%) & Ant. reg. \\ 
  \hline
Tidligere slag & 29.6 & 23.5 & 314 \\ 
  Tidligere TIA & 13.2 & 8.2 & 310 \\ 
  Tidligere hjerteinfarkt & 6.7 & 19.1 & 312 \\ 
  Røyking før hjerneslaget & 17.6 & 23.6 & 290 \\ 
  Diabetes & 20.4 & 19.5 & 311 \\ 
  Behandling høyt BT & 41.8 & 36.4 & 231 \\ 
  Lipidsenkende behandling & 25.5 & 32.9 & 314 \\ 
  Atrieflimmer & 27.8 & 26.4 & 310 \\ 
   \hline
\end{tabular}
\caption{Risikofaktorer før hjerneslaget, St. Olav. Lipidsenkende behandling er registrert 
		fra 1.januar 2014.} 
\label{tab:Risiko1}
\end{table}
% latex table generated in R 3.2.3 by xtable 1.8-2 package
% Tue Mar 08 13:14:17 2016
\begin{table}[ht]
\centering
\begin{tabular}{lrrr}
  \hline
 & Kvinner (\%) & Menn (\%) & Ant. reg. \\ 
  \hline
Tidligere slag & 24.3 & 22.5 & 4615 \\ 
  Tidligere TIA & 12.1 & 10.8 & 4489 \\ 
  Tidligere hjerteinfarkt & 11.2 & 20.7 & 4601 \\ 
  Røyking før hjerneslaget & 23.4 & 28.0 & 3783 \\ 
  Diabetes & 17.2 & 20.6 & 4613 \\ 
  Behandling høyt BT & 43.8 & 41.3 & 3478 \\ 
  Lipidsenkende behandling & 29.6 & 38.2 & 4610 \\ 
  Atrieflimmer & 27.4 & 25.3 & 4587 \\ 
   \hline
\end{tabular}
\caption{Risikofaktorer før hjerneslaget, hele landet. Lipidsenkende behandling er registrert 
		fra 1.januar 2014.} 
\label{tab:Risiko2}
\end{table}



Ved angivelse av hvilke medikamenter pasientene har fått  angir oppgitte enkeltmedikamenter 
andel pasienter i medikamentgruppa 
som har fått kun dette medikamentet. Eksempelvis i tabell \ref{tab:MedBehPre1} vil andelen ASA 
angi andelen av de som har fått platehemmende behandling kun med ASA. Det betyr at andelen 
ukjente i en eller flere av de tre variablene som utgjør platehemmende behandling, vil spille 
en stor rolle for muligheten til å angi hvor stor andel som kun har fått for eksempel ASA.

% latex table generated in R 3.2.3 by xtable 1.8-2 package
% Tue Mar 08 13:14:17 2016
\begin{table}[ht]
\centering
\begin{tabular}{lrr}
  \hline
 & Ja (\%) & Ukjent (\%) \\ 
  \hline
Platehemmende beh. & 33.2 & 0.5 \\ 
  ...ASA & 76.8 &  \\ 
  ...Dipyridamol & 1.4 &  \\ 
  ...Klopidogrel & 7.2 &  \\ 
  ...ASA+Dipyridamol & 14.5 &  \\ 
  ...ASA+Klopidogrel & 0.0 &  \\ 
  Antikoagulerende & 11.5 & 0.0 \\ 
  ...Antikoagulasjon m/Warfarin & 79.2 &  \\ 
  ...Andre perorale antikoag. & 20.8 &  \\ 
  BT-senkende & 54.3 & 4.3 \\ 
  ...Diuretica & 10.6 &  \\ 
  ...ACE-hemmer & 5.3 &  \\ 
  ...A2-antagonist & 16.8 &  \\ 
  ...Betablokker & 15.9 &  \\ 
  ...Kalsiumantagonist & 10.6 &  \\ 
  ...Komb. 2 BT-senkende & 29.2 &  \\ 
  ...Komb. 3 BT-senkende & 8.8 &  \\ 
  ...Komb. 4 BT-senkende & 1.8 &  \\ 
  ...Komb. alle BT-senkende & 0.0 &  \\ 
  Statin-Lipidsenkende & 27.4 & 0.0 \\ 
   \hline
\end{tabular}
\caption{Medikamentell behandling før hjerneslaget. Innleggelser fra og med 1.januar 2014. 
		Alle diagnoser, St. Olav (N=208).} 
\label{tab:MedBehPre1}
\end{table}
% latex table generated in R 3.2.3 by xtable 1.8-2 package
% Tue Mar 08 13:14:17 2016
\begin{table}[ht]
\centering
\begin{tabular}{lrr}
  \hline
 & Ja (\%) & Ukjent (\%) \\ 
  \hline
Platehemmende beh. & 40.7 & 0.9 \\ 
  ...ASA & 73.9 &  \\ 
  ...Dipyridamol & 2.7 &  \\ 
  ...Klopidogrel & 4.9 &  \\ 
  ...ASA+Dipyridamol & 12.8 &  \\ 
  ...ASA+Klopidogrel & 4.8 &  \\ 
  Antikoagulerende & 13.2 & 0.8 \\ 
  ...Antikoagulasjon m/Warfarin & 65.8 &  \\ 
  ...Andre perorale antikoag. & 32.3 &  \\ 
  BT-senkende & 59.5 & 2.3 \\ 
  ...Diuretica & 7.6 &  \\ 
  ...ACE-hemmer & 4.7 &  \\ 
  ...A2-antagonist & 10.2 &  \\ 
  ...Betablokker & 17.0 &  \\ 
  ...Kalsiumantagonist & 7.2 &  \\ 
  ...Komb. 2 BT-senkende & 33.6 &  \\ 
  ...Komb. 3 BT-senkende & 15.7 &  \\ 
  ...Komb. 4 BT-senkende & 3.0 &  \\ 
  ...Komb. alle BT-senkende & 0.3 &  \\ 
  Statin-Lipidsenkende & 34.0 & 0.9 \\ 
   \hline
\end{tabular}
\caption{Medikamentell behandling før hjerneslaget. Innleggelser fra og med 1.januar 2014. 
		Alle diagnoser, hele landet (N=3235).} 
\label{tab:MedBehPre2}
\end{table}

		
% latex table generated in R 3.2.3 by xtable 1.8-2 package
% Tue Mar 08 13:14:17 2016
\begin{table}[ht]
\centering
\begin{tabular}{lrr}
  \hline
 & Ja (\%) & Ukjent (\%) \\ 
  \hline
Platehemmende beh. & 10.0 & 0.0 \\ 
  ...ASA & 100.0 &  \\ 
  ...Dipyridamol & 0.0 &  \\ 
  ...Klopidogrel & 0.0 &  \\ 
  ...ASA+Dipyridamol & 0.0 &  \\ 
  ...ASA+Klopidogrel & 0.0 &  \\ 
  Antikoagulerende & 15.0 & 0.0 \\ 
  ...Antikoagulasjon m/Warfarin & 100.0 &  \\ 
  ...Andre perorale antikoag. & 0.0 &  \\ 
   \hline
\end{tabular}
\caption{Medikamentell behandling før hjerneslaget for pasienter med hjerneblødning (I61), St. Olav. N=20. (Innleggelser fra og med 1.januar 2014.)} 
\label{tab:MedBehPre1Diag1}
\end{table}
% latex table generated in R 3.2.3 by xtable 1.8-2 package
% Tue Mar 08 13:14:17 2016
\begin{table}[ht]
\centering
\begin{tabular}{lrr}
  \hline
 & Ja (\%) & Ukjent (\%) \\ 
  \hline
Platehemmende beh. & 29.9 & 1.5 \\ 
  ...ASA & 75.6 &  \\ 
  ...Dipyridamol & 0.8 &  \\ 
  ...Klopidogrel & 4.1 &  \\ 
  ...ASA+Dipyridamol & 12.2 &  \\ 
  ...ASA+Klopidogrel & 5.7 &  \\ 
  Antikoagulerende & 22.6 & 0.7 \\ 
  ...Antikoagulasjon m/Warfarin & 66.7 &  \\ 
  ...Andre perorale antikoag. & 30.1 &  \\ 
   \hline
\end{tabular}
\caption{Medikamentell behandling før hjerneslaget for pasienter med hjerneblødning (I61), hele landet. N=411. (Innleggelser fra og med 1.januar 2014.)} 
\label{tab:MedBehPre2Diag1}
\end{table}
% latex table generated in R 3.2.3 by xtable 1.8-2 package
% Tue Mar 08 13:14:17 2016
\begin{table}[ht]
\centering
\begin{tabular}{lrr}
  \hline
 & Ja (\%) & Ukjent (\%) \\ 
  \hline
Platehemmende beh. & 36.0 & 0.5 \\ 
  ...ASA & 76.1 &  \\ 
  ...Dipyridamol & 1.5 &  \\ 
  ...Klopidogrel & 7.5 &  \\ 
  ...ASA+Dipyridamol & 14.9 &  \\ 
  ...ASA+Klopidogrel & 0.0 &  \\ 
  Antikoagulerende & 11.3 & 0.0 \\ 
  ...Antikoagulasjon m/Warfarin & 76.2 &  \\ 
  ...Andre perorale antikoag. & 23.8 &  \\ 
   \hline
\end{tabular}
\caption{Medikamentell behandling før hjerneslaget for pasienter med hjerneinfarkt (I63), St. Olav. N=186. (Innleggelser fra og med 1.januar 2014.)} 
\label{tab:MedBehPre1Diag2}
\end{table}
% latex table generated in R 3.2.3 by xtable 1.8-2 package
% Tue Mar 08 13:14:17 2016
\begin{table}[ht]
\centering
\begin{tabular}{lrr}
  \hline
 & Ja (\%) & Ukjent (\%) \\ 
  \hline
Platehemmende beh. & 42.1 & 0.7 \\ 
  ...ASA & 73.6 &  \\ 
  ...Dipyridamol & 3.0 &  \\ 
  ...Klopidogrel & 5.1 &  \\ 
  ...ASA+Dipyridamol & 12.8 &  \\ 
  ...ASA+Klopidogrel & 4.7 &  \\ 
  Antikoagulerende & 11.8 & 1.3 \\ 
  ...Antikoagulasjon m/Warfarin & 65.6 &  \\ 
  ...Andre perorale antikoag. & 33.1 &  \\ 
   \hline
\end{tabular}
\caption{Medikamentell behandling før hjerneslaget for pasienter med hjerneinfarkt (I63), hele landet. N=2768. (Innleggelser fra og med 1.januar 2014.)} 
\label{tab:MedBehPre2Diag2}
\end{table}





\clearpage
\section{Akuttfasen}

Dette avsnittet inneholder resultater fra behandling i akuttfasen. Dette omfatter:
\begin{itemize}
	\item Tid fra symptomdebut til innleggelse
	\item Tid fra innleggelse til trombolyse
	\item Symptomer
	\item Reperfusjonsbehandling
	\item Utredning, bildediagnostikk og hjerterytmeregistrering
\end{itemize}




\begin{figure}[ht]
\centering \includegraphics[width=0.75\textwidth]{FigTidSymptInnlegg.pdf}
\caption{\label{fig:TidSymptInnlegg} Tid fra de første symptomene på slag til innleggelse.}
\end{figure}

\begin{figure}[ht]
\centering \includegraphics[width=0.75\textwidth]{FigTidInnTromb.pdf}
\caption{\label{fig:TidInnTromb}Tid fra innleggelse til trombolyse}
\end{figure}

\begin{figure}[ht]
\centering \includegraphics[width=0.75\textwidth]{FigBevGrInn.pdf}
\caption{\label{fig:BevissthetsgradInn} Bevissthetsgrad ved innleggelse.}
\end{figure}

\begin{figure}[ht]
\centering \includegraphics[width=0.75\textwidth]{FigNIHSSinn.pdf}
\caption{\label{fig:NIHSSinnkomst} NIHSS score ved innleggelse (alle diagnoser).}
\end{figure}

\begin{figure}[ht]
\centering \includegraphics[width=0.75\textwidth]{FigAvdForst.pdf}
\caption{\label{fig:AvdForstInnlagt} Ved hvilken type avdeling ble pasientene først innlagt.}
\end{figure}


Tabell \ref{tab:AkuttbehI63} viser oversikt over akuttbehandling av hjerneinfarkt (diagnose I63).
Andelene er beregnet ift. sikre svar (dvs. bare "ja/nei" ).
Andelen ukjente er lav så om disse var tatt med, ville det ikke påvirke resultatene i særlig grad.


% latex table generated in R 3.2.3 by xtable 1.8-2 package
% Tue Mar 08 13:14:18 2016
\begin{table}[ht]
\centering
\begin{tabular}{lrrrr}
  \hline
 & Ant.Eget & Andel(\%) & Ant.Landet & Andel(\%) \\ 
  \hline
Trombolyse & 45 & 16.2 & 646 & 16.3 \\ 
  Trombolyse $<$=80år & 32 & 19.6 & 443 & 18.3 \\ 
  Trombolyse, $>$80år & 13 & 11.3 & 203 & 13.2 \\ 
  Hjernebl m/forverring innen 36t & 3 & 7.0 & 27 & 4.4 \\ 
  Trombektomi & 5 & 1.8 & 59 & 1.5 \\ 
  Trombektomi, $<$=80år & 5 & 3.1 & 55 & 2.3 \\ 
   \hline
\end{tabular}
\caption{Reperfusjonsbehandling av hjerneinfarkt (I63), St. Olav og hele landet.} 
\label{tab:AkuttbehI63}
\end{table}




%\begin{figure}[ht]
%<<FigNIHSSpreTrombolyse, echo=FALSE, message=FALSE, eval=T>>=
%	FigAndeler(RegData=SlagData, valgtVar='NIHSSpreTrombolyse', datoFra=datoFra, 
%		datoTil=datoTil, minald=minald, maxald=maxald, erMann=erMann, diagnose=diagnose, innl4t=innl4t, 
%		NIHSSinn=NIHSSinn, reshID=reshID, enhetsUtvalg, hentData=0, preprosess=preprosess, outfile=outfile)
%@
%\caption{\label{fig:NIHSSpreTrombolyse} NIHSS score før trombolyse.}
%\end{figure}

%\begin{figure}[ht]
%<<FigNIHSSetterTrombolyse, echo=FALSE, message=FALSE, eval=T>>=
%	FigAndeler(RegData=SlagData, valgtVar='NIHSSetterTrombolyse', datoFra=datoFra, 
%		datoTil=datoTil, minald=minald, maxald=maxald, erMann=erMann, diagnose=diagnose, innl4t=innl4t, 
%		NIHSSinn=NIHSSinn, reshID=reshID, enhetsUtvalg, hentData=0, preprosess=preprosess, outfile=outfile)
%@
%\caption{\label{fig:NIHSSetterTrombolyse} NIHSS score etter trombolyse.}
%\end{figure}





% latex table generated in R 3.2.3 by xtable 1.8-2 package
% Tue Mar 08 13:14:18 2016
\begin{table}[ht]
\centering
\begin{tabular}{lrr}
  \hline
 & Ja (\%) & Ukjent (\%) \\ 
  \hline
CT/MR innen 12t & 97.8 & 0.6 \\ 
  Fokale utf m/pos. bildediag. & 48.3 & 0.0 \\ 
  Hjerneblødning (I61) & 10.9 &  \\ 
  Hjerneinfarkt (I63) & 89.1 &  \\ 
  Uspes. slag (I64) & 0.0 &  \\ 
  Svelgfunksjon vurdert & 86.2 & 10.1 \\ 
  Mobilisert ila. 24t & 78.2 & 4.9 \\ 
  Tverrfaglig vurdering & 90.9 & 2.1 \\ 
   \hline
\end{tabular}
\caption{Utførte undersøkelser og tiltak, St. Olav. 
		For Fokale utfall er det kun krysset av på Ja og Nei, og ikke krysset av på ukjente. 
		I tabellen er derfor manglende registreringer angitt som ukjent.} 
\label{tab:KritSupp1}
\end{table}
% latex table generated in R 3.2.3 by xtable 1.8-2 package
% Tue Mar 08 13:14:18 2016
\begin{table}[ht]
\centering
\begin{tabular}{lrr}
  \hline
 & Ja (\%) & Ukjent (\%) \\ 
  \hline
CT/MR innen 12t & 95.6 & 1.2 \\ 
  Fokale utf m/pos. bildediag. & 45.1 & 0.0 \\ 
  Hjerneblødning (I61) & 13.1 &  \\ 
  Hjerneinfarkt (I63) & 86.9 &  \\ 
  Uspes. slag (I64) & 0.0 &  \\ 
  Svelgfunksjon vurdert & 79.0 & 8.7 \\ 
  Mobilisert ila. 24t & 75.3 & 5.2 \\ 
  Tverrfaglig vurdering & 82.9 & 2.5 \\ 
   \hline
\end{tabular}
\caption{Utførte undersøkelser og tiltak, hele landet. 
		For Fokale utfall er det kun krysset av på Ja og Nei, og ikke krysset av på ukjente. 
		I tabellen er derfor manglende registreringer angitt som ukjent.} 
\label{tab:KritSupp2}
\end{table}



\begin{figure}[ht]
\centering \includegraphics[width=0.75\textwidth]{FigNIHSSppTrombolyseEget.pdf}
\caption{\label{fig:NIHSSendrTrombolyse} NIHSS score før og etter trombolyse, St. Olav.}
\end{figure}

\begin{figure}[ht]
\centering \includegraphics[width=0.75\textwidth]{FigNIHSSendrTrombolyse.pdf}
\caption{\label{fig:NIHSSendrTrombolyse} Endring i NIHSS fra før til etter trombolyse.}
\end{figure}

\begin{figure}[ht]
\centering \includegraphics[width=0.75\textwidth]{FigBildediagnostikkHjerne.pdf}
\caption{\label{fig:BildediagnostikkHjerne} Bildediagnostikk av hjerneslaget.}
\end{figure}

\begin{figure}[ht]
\centering \includegraphics[width=0.75\textwidth]{FigBildediagEksKar.pdf}
\caption{\label{fig:BildediagEksKar} Bildediagnostikk av ekstrakranielle kar.}
\end{figure}

\begin{figure}[ht]
\centering \includegraphics[width=0.75\textwidth]{FigBildediagIntraKar.pdf}
\caption{\label{fig:BildediagIntraKar} Bildediagnostikk av intrakranielle kar.}
\end{figure}

\begin{figure}[ht]
\centering \includegraphics[width=0.75\textwidth]{FigHjerterytme.pdf}
\caption{\label{fig:Hjerterytme} Registrering av hjerterytme.}
\end{figure}

\begin{figure}[ht]
\centering \includegraphics[width=0.75\textwidth]{FigBildediagnostikkHjerte.pdf}
\caption{\label{fig:BildediagnostikkHjerte} Bildediagnostikk av hjertet.}
\end{figure}


\clearpage
	

	
\section{Sekundær profylakse}
	

% latex table generated in R 3.2.3 by xtable 1.8-2 package
% Tue Mar 08 13:14:19 2016
\begin{table}[ht]
\centering
\begin{tabular}{lrr}
  \hline
 & Ja (\%) & Ukjent (\%) \\ 
  \hline
Antitrombotisk beh. & 96.0 & 0.0 \\ 
  Platehemmende beh. & 79.2 & 0.0 \\ 
  ...ASA & 48.9 &  \\ 
  ...Dipyridamol & 0.0 &  \\ 
  ...Klopidogrel & 14.6 &  \\ 
  ...ASA+Dipyridamol & 29.9 &  \\ 
  ...ASA+Klopidogrel & 6.6 &  \\ 
  Antikoagulerende & 22.0 & 0.0 \\ 
  ...Antikoagulasjon m/Warfarin & 73.7 &  \\ 
  ...Andre perorale antikoag. & 26.3 &  \\ 
   \hline
\end{tabular}
\caption{Platehemmende og antikoagulerende behandling ved utskriving (kun levende). 
		Diagnose hjerneinfarkt (I63), St. Olav, N=173. (Innleggelser fra og med 1. januar 2014.)} 
\label{tab:MedBehPAUt1}
\end{table}
% latex table generated in R 3.2.3 by xtable 1.8-2 package
% Tue Mar 08 13:14:19 2016
\begin{table}[ht]
\centering
\begin{tabular}{lrr}
  \hline
 & Ja (\%) & Ukjent (\%) \\ 
  \hline
Antitrombotisk beh. & 91.4 & 1.1 \\ 
  Platehemmende beh. & 72.2 & 1.1 \\ 
  ...ASA & 34.7 &  \\ 
  ...Dipyridamol & 4.7 &  \\ 
  ...Klopidogrel & 11.4 &  \\ 
  ...ASA+Dipyridamol & 40.5 &  \\ 
  ...ASA+Klopidogrel & 8.0 &  \\ 
  Antikoagulerende & 24.3 & 1.3 \\ 
  ...Antikoagulasjon m/Warfarin & 38.6 &  \\ 
  ...Andre perorale antikoag. & 59.9 &  \\ 
   \hline
\end{tabular}
\caption{Platehemmende og antikoagulerende behandling ved utskriving (kun levende). 
		Diagnose hjerneinfarkt (I63), hele landet, N=2628. (Innleggelser fra og med 1. januar 2014.)} 
\label{tab:MedBehPAUt2}
\end{table}
% latex table generated in R 3.2.3 by xtable 1.8-2 package
% Tue Mar 08 13:14:19 2016
\begin{table}[ht]
\centering
\begin{tabular}{lrr}
  \hline
 & Andel(\%):  Eget & Hele landet \\ 
  \hline
Antikoagulerende, $<$=80år & 89.3 & 77.7 \\ 
  ... Warfarin & 72.0 & 37.4 \\ 
  ... andre perorale antikoag. & 28.0 & 60.6 \\ 
  Antikoagkoagulerende,  81+år & 56.1 & 63.0 \\ 
  ... Warfarin  & 78.3 & 49.0 \\ 
  ... andre  & 21.7 & 50.4 \\ 
   \hline
\end{tabular}
\caption{Antikoagulasjonsbehandling ved utskriving av pasienter med atrieflimmer som har  
			overlevd hjerneinfarkt (I63).} 
\label{tab:MedBehDivUt}
\end{table}
% latex table generated in R 3.2.3 by xtable 1.8-2 package
% Tue Mar 08 13:14:19 2016
\begin{table}[ht]
\centering
\begin{tabular}{lrr}
  \hline
 & Ja (\%) & Ukjent (\%) \\ 
  \hline
BT-senkende & 59.0 & 2.5 \\ 
  ...Diuretica & 7.0 &  \\ 
  ...ACE-hemmer & 8.6 &  \\ 
  ...A2-antagonist & 14.0 &  \\ 
  ...Betablokker & 17.7 &  \\ 
  ...Kalsiumantagonist & 10.8 &  \\ 
  ...Komb. 2 BT-senkende & 31.2 &  \\ 
  ...Komb. 3 BT-senkende & 10.2 &  \\ 
  ...Komb. 4 BT-senkende & 0.5 &  \\ 
  ...Komb. alle BT-senkende & 0.0 &  \\ 
  Statin-Lipidsenkende & 57.1 &  \\ 
   \hline
\end{tabular}
\caption{Blodtrykkssenkende behandling og statiner ved utskriving (kun levende). 
		Alle diagnoser, St. Olav, N=315.} 
\label{tab:MedBehBTSUt1}
\end{table}
% latex table generated in R 3.2.3 by xtable 1.8-2 package
% Tue Mar 08 13:14:19 2016
\begin{table}[ht]
\centering
\begin{tabular}{lrr}
  \hline
 & Ja (\%) & Ukjent (\%) \\ 
  \hline
BT-senkende & 61.4 & 2.0 \\ 
  ...Diuretica & 6.0 &  \\ 
  ...ACE-hemmer & 5.4 &  \\ 
  ...A2-antagonist & 11.3 &  \\ 
  ...Betablokker & 14.4 &  \\ 
  ...Kalsiumantagonist & 9.0 &  \\ 
  ...Komb. 2 BT-senkende & 32.6 &  \\ 
  ...Komb. 3 BT-senkende & 16.6 &  \\ 
  ...Komb. 4 BT-senkende & 3.5 &  \\ 
  ...Komb. alle BT-senkende & 0.1 &  \\ 
  Statin-Lipidsenkende & 61.0 & 1.1 \\ 
   \hline
\end{tabular}
\caption{Blodtrykkssenkende behandling og statiner ved utskriving (kun levende). 
		Alle diagnoser, hele landet, N=4652.} 
\label{tab:MedBehBTSUt2}
\end{table}


\clearpage
\section{Utskriving}

Dette kapitlet inneholder informasjon om:
\begin{itemize}
	\item Liggetid
	\item Utskrevet fra
	\item Utskrevet til
\end{itemize}




\begin{figure}[ht]
\centering \includegraphics[width=0.75\textwidth]{FigDagerInnl.pdf}
\caption{\label{fig:AntDagerInnl} Liggetid i forbindelse med slaget.}
\end{figure}

\begin{figure}[ht]
\centering \includegraphics[width=0.75\textwidth]{FigAvdUtskrFraHvilken.pdf}
\caption{\label{fig:AvdUtskrFraHvilken} Fra hvilken avdeling ble pasienten utskrevet?}
\end{figure}


\begin{figure}[ht]
\centering \includegraphics[width=0.75\textwidth]{FigUtskrTil.pdf}
\caption{\label{fig:UtskrTil} Hva ble pasientene skrevet ut til?}
\end{figure}


\clearpage
\section{Oppfølging}

Dette kapitlet inneholder resultater knyttet til:
\begin{itemize}
	\item Boligforhold/bosituasjon
%	\item Røykestatus
	\item Selvhjulpenhet (toalett, forflytning, påkledning)
	\item Bilkjøring
	\item Tilfredshet
\end{itemize}






\begin{figure}[ht]
\centering \includegraphics[width=0.75\textwidth]{FigBoligforh.pdf}
\caption{\label{fig:Boligforh} Boligforhold før og 3 måneder etter slaget.}
\end{figure}

\begin{figure}[ht]
\centering \includegraphics[width=0.75\textwidth]{FigBositPrePost.pdf}
\caption{\label{fig:BositPrePost} Bosituasjon ved innleggelse og 3 måneder etter slaget.}
\end{figure}


%\begin{figure}[ht]
%<<FigRoykestatusPrePost, echo=FALSE, message=FALSE, eval=T>>=
%	FigPrePost(RegData=SlagData, valgtVar='Roykestatus', datoFra=datoFra, 
%		datoTil=datoTil, minald=minald, maxald=maxald, erMann=erMann, diagnose=diagnose, 
%		innl4t=innl4t, NIHSSinn=NIHSSinn, reshID=reshID, enhetsUtvalg=enhetsUtvalg, preprosess=preprosess, outfile=outfile)
%@
%\caption{\label{fig:RoykPrePost} Røykestatus før og 3 måneder etter slaget.}
%\end{figure}

\begin{figure}[ht]
\centering \includegraphics[width=0.75\textwidth]{FigToalettPrePost.pdf}
\caption{\label{fig:ToalettPrePost} Selvhjulpenhet ved toalettbesøk før og 3 måneder etter slaget.}
\end{figure}

\begin{figure}[ht]
\centering \includegraphics[width=0.75\textwidth]{FigForflytningPrePost.pdf}
\caption{\label{fig:ForflytningPrePost} Forflytningsevne før og 3 måneder etter slaget.}
\end{figure}

\begin{figure}[ht]
\centering \includegraphics[width=0.75\textwidth]{FigPaakledningPrePost.pdf}
\caption{\label{fig:PaakledningPrePost} Evne til av-/påkledning før og 3 måneder etter slaget.}
\end{figure}


\begin{figure}[ht]
\centering \includegraphics[width=0.75\textwidth]{FigBilkjoringPrePost.pdf}
\caption{\label{fig:BilkjoringPrePost} Evne til bilkjøring før og 3 måneder etter slaget.}
\end{figure}


\begin{figure}[ht]
\centering \includegraphics[width=0.75\textwidth]{FigTilfredshet.pdf}
\caption{\label{fig:Tilfredshet} Er pasienten like fornøyd med tilværelsen etter 
		hjerneslaget som før?}
\end{figure}

\clearpage
%\input{libkatTexansvarsforhold}

\end{document}
