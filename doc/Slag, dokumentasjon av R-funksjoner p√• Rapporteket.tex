\documentclass [norsk,a4paper,twoside]{article}
\addtolength{\hoffset}{-0.5cm}
\addtolength{\textwidth}{1cm}
\addtolength{\voffset}{-1cm}
\addtolength{\textheight}{2cm}


%for nice looking tabs
\usepackage{booktabs}

\usepackage[norsk]{babel}
\usepackage[utf8x]{inputenc}
\usepackage{textcomp}
\usepackage{fancyhdr}
\pagestyle{fancy}
\usepackage{amsmath}
\usepackage{rotating} %add rotating for plain tables
\usepackage{pdflscape} %add rotating/landcape for pdf

%add rotating for plain tables
\usepackage{rotating}

%add rotating/landcape for pdf
\usepackage{pdflscape}

%bytte font
\renewcommand{\familydefault}{\sfdefault}

%setter grå skrift fremfort sort
\usepackage{xcolor}
\usepackage{graphicx}

%Offentliggjøringsfargene 1-6, lys-mørk, benytter 2-5
\definecolor{OffBlaa2}{rgb}{0.42, 0.68, 0.84}	%107/255, 174/255, 214/255}
\definecolor{OffBlaa3}{rgb}{0.26, 0.57, 0.78}	%66/255, 146/255, 198/255}
\definecolor{OffBlaa4}{rgb}{ 0.13 0.44 0.71}	%33/255, 113/255, 181/255}
\definecolor{OffBlaa5}{rgb}{0.03 0.27 0.58}		%8/255,  69/255, 148/255}
%\definecolor{SKDE}{rgb}{0,0.32,0.61}
%\definecolor{moerkgraa}{rgb}{0.25,0.25,0.25}
%\color{moerkgraa}

\usepackage[pdftex, colorlinks, linkcolor=OffBlaa3, urlcolor=OffBlaa3]{hyperref}
\usepackage{enumitem}
\setlist[itemize]{noitemsep}	%, nolistsep
%Kan også bruke itemsep=... osv.

%bytte overskrifter 
\usepackage[compact]{titlesec}
\titleformat{\section} {\vspace*{10pt}\color{OffBlaa5}\normalfont\Large\bfseries} {\thesection}{}{}
\titleformat{\subsection} {\color{OffBlaa3}\normalfont\large\bfseries} {\thesection}{}{}

%topptekst og vertikal sidenummer
\fancyhead{}\fancyfoot{}  % clear all fields
\fancyheadoffset[LO, RE]{3cm}
\fancyfootoffset[LO]{1.5cm}
\fancyfootoffset[RE]{1.5cm}
%Stripe øverst på sida med registerets navn
\fancyhead[LO]{\colorbox{OffBlaa5}{\textcolor{white}{\hspace*{2cm}\scshape\small Norsk Intensivregister}}} %Lengde på stripa
\fancyfoot[LO]{\colorbox{OffBlaa5}{\textcolor{white}{\scshape\small\thepage}}} 
\fancyfoot[RE]{\colorbox{OffBlaa5}{\textcolor{white}{\scshape\small\thepage}}}
\renewcommand{\headrulewidth}{0pt} %\iffloatpage{0pt}{0.4pt}
 \renewcommand{\footrulewidth}{0pt}
%evt. horisontal sidenummer
\fancyfoot[LO]{\colorbox{OffBlaa5}{\textcolor{white}{\hspace*{2cm} \small \thepage}}} \fancyfootoffset[LO]{4.4cm}
\fancyfoot[RE]{\hspace*{2cm}\colorbox{OffBlaa5}{\textcolor{white}{\small \thepage \hspace*{3cm}}}}  \fancyfootoffset[RE]{5.3cm}
\setcounter{secnumdepth}{-1} 


\begin{document}
\begin{titlepage}
\newcommand{\HRule}{\rule{\linewidth}{0.5mm}} % Defines a new command for the horizontal lines, change thickness here
\center % Center everything on the page
%	TITLE SECTION
\HRule \\[0.4cm]
{ \huge \bfseries Dokumentasjon for R-funksjoner i HJERNESLAG}\\[0.4cm] % Title of your document
\HRule \\[1.5cm]
 
%	AUTHOR SECTION
%\author{Lena Ringstad Olsen, SKDE}
%\Large \emph{Author:}\\
\Large Utarbeidet av: \\
\huge{Lena Ringstad Olsen, SKDE}\\[3cm] % Your name

%	DATE SECTION
{\Large \today}\\[2cm] % Date, change the \today to a set date if you want to be precise

\vspace{5cm}

%	LOGO SECTION
%\includegraphics[height=2cm]{LogoNIR.jpg}\\[1cm] % Include a department/university logo - this will require the graphicx package
 
 \vfill % Fill the rest of the page with whitespace
\end{titlepage}


\section{Register og data}
Hjerneslagregisteret har et hovedskjema og et oppfølgingsskjema som fylles ut 3 måneder etter hjerneslaget. På Rapporteket er det gjort ei omstrukturering av dataene, dvs. at hvert opphold er ei linje i datasettet. (I databasen er hver registrering ei linje, dvs. at hovedskjema og oppfølgingsskjema tilhørende samme innleggelse kommer på to rader.)
Resultatene baserer seg på registreringer fra og med 1.januar 2013. Etter ønske fra registerledelsen er registreringer før den tid, ikke tilgjengliggjort på Rapporteket.

\section{Utvalg av data/inputparametre}

Generelt er det mulig å gjøre utvalg på: 
\begin{itemize}
\item Tidsperiode for innleggelser: f.o.m. – t.o.m., alltid angitt 
\item Alder  min – maks, standard:alle [minald,maksald]
\item Kjønn [erMann]
	\subitem 0: kvinner
	\subitem 1: menn
	\subitem alt annet enn 0 eller 1: begge (standard)]
\item diagnose 
	\subitem	1- Infarkt(I61)
	\subitem	2- Blødning(I63)
	\subitem	3-Udefinert(I64) 
	\subitem	standard: '' (alt annet)
\item innlagt innen 4 timer [innl4t]
%	\begin{itemize}
	\subitem	0: nei, 
	\subitem	1: ja, 
	\subitem	standard: '' (alt annet)
%	\end{itemize}
\item NIHSS ved innleggelse [NIHSSinn]
%	\begin{itemize}
	\subitem	1: 0-5, 	
	\subitem	2: 6-10, 
	\subitem	3: 11-15, 
	\subitem	4: 16-20, 
	\subitem	5: 21-25, 
	\subitem	standard: '' (alt annet)
	\subitem	Velges denne, blir registreringer hvor NIHSS ikke er utført, tatt bort (selvsagt)	
%	\end{itemize}
\item Hvilket nivå av data man ønsker å se resultater for [enhetsUtvalg]
	\subitem 0: Hele landet
	\subitem 1: Egen enhet mot resten av landet (standard)
	\subitem 2: Egen enhet
\item Variabel: Hvilken variabel figuren skal framstille [valgtVar]
\end{itemize} 
%\vspace{1cm}
	


Alle utvalg er ikke aktuelle for alle figurer. I tillegg kommer det mer spesifikke valg som f.eks. om en figur skal vise gjennomsnitt (standard) eller median, der dette er akutelt.

I tillegg kommer noen "tekniske" inputparametre:
\begin{itemize}
	\item[RegData] Registerdata, dvs. ei datamatrise med variable og registreringer
	\item[libkat] sti til hvor man finner bibliotekfunksjonene de enkelte figurfunksjonene benytter
	\item[outfile] angivelse av navn/type figurfil figuren skal skrives til (eks Alder.png)
	\item[reshID] ID for enheten som ber om data. Dette bestemmer hvilken enhet resultatet vises for
\end{itemize}


\section{Figurfunksjoner}


\subsection{Funksjon: SlagFigAntStabel.R}
Søylediagram som viser antall registreringer per måned, siste 12 hele måneder fra valgt sluttdato av hovedskjema, oppfølgingsskjema og døde.

\begin{figure}
	\centering
	\includegraphics[width=0.7\linewidth]{../trunk/RAntStabel/Registreringer}
	\caption{}
	\label{fig:Registreringer}
\end{figure}

FigAntReg  <- function(RegData, datoTil='2050-12-31', minald=0, maxald=130, erMann='', diagnose='', innl4t='', NIHSSinn='', libkat, outfile='', reshID, enhetsUtvalg=2)



\subsection{Funksjon: SlagFigAndeler.R (Fordelinger)}
Funksjonen viser en figur med fordelinga til valgt variabel (spesifisert ved parameteren "valgtVar"). Andel observasjoner i hver kategori vises horisontalt eller vertikalt. Velger man
å vise egen avdeling mot resten av landet (enhetsUtvalg=1), kommer resultater for egen avdeling som søyler, mens andelene for resten av landet vises som en "prikk".

\begin{figure}
\centering
\includegraphics[width=0.7\linewidth]{../trunk/RAndeler/Alder.pdf}
\caption{Aldersfordeling for egen enhet mot resten av landet.}
\label{fig:Alder}
\end{figure}

\subsubsection{Funksjonsdefinisjon}
FigAndeler  <- function(RegData, valgtVar, datoFra='2012-04-01', datoTil='2050-12-31', 
minald=0, maxald=130, erMann='', diagnose='', innl4t='', NIHSSinn='', libkat, outfile='', 
reshID, enhetsUtvalg=1)

Hvilken variabel man vil se fordelinga til, velger man ved å spesifisere parameteren "valgtVar". Figuren er definert for følgende variable:\\

Alder, Transportmetode,
AntDagerInnl, TidSymptInnlegg, TidSymptTrombolyse, TidInnleggTrombolyse
NIHSSinnkomst, NIHSSendrTrombolyse, NIHSSendrTrombektomi
NIHSSpreTrombolyse','NIHSSetterTrombolyse','NIHSSpreTrombektomi', 'NIHSSetterTrombektomi'
Slagdiagnose, BildediagnostikkHjerne, BildediagnostikkHjerte,
BildediagnostikkIntraraniell, BildediagnostikkEkstrakranKar, RegistreringHjerterytme
BoligforholdPre, RoykerPre, Royker3mnd, 'MRSPre', 'MRS3mnd', Tilfredshet,
SivilstatusPre, BevissthetsgradInnleggelse, 
AvdForstInnlagtHvilken, AvdUtskrFraHvilken, UtskrTil
FokaleUtf, FokaleUtfAndre,



\subsection{Funksjon: SlagFigAndelerKvalInd.R}
Funksjonen viser oversikt over en rekke kvalitetsindikatorer, angitt med hvor stor andel av pasientene som har fått den aktuelle behandlinga. Figuren er i prinsippet identisk med FigAndeler, men mange spesielle utvalg tvang fram mye "hardkoding" og figuren ble derfor skilt ut i egen funksjon. 

FigAndelerKvalInd  <- function(RegData, datoFra='2012-04-01', datoTil='2050-12-31', 
erMann='', NIHSSinn='', libkat, outfile='', enhetsUtvalg=1, reshID)



\subsection{Funksjon: SlagFigPrePost.R}
Søylediagram som fordeling av variable målt både før og etter hjerneslaget, evt. sammenlignet med resten av landet. 


FigPrePost  <- function(RegData, valgtVar, datoFra='2012-04-01', datoTil='2050-12-31', 
minald=0, maxald=130, erMann='', diagnose='', innl4t='', NIHSSinn='', libkat, outfile='', 
reshID, enhetsUtvalg=1)

Figuren er definert for følgende variable: \\
 "Bilkjoring"       "Boligforh"        "Bosituasjon"      "Forflytning"      "MRS"             
 "NIHSSTrombektomi" "NIHSSTrombolyse"  "Paakledning"      "Toalett"          "Yrkesaktiv" 



\subsection{Funksjon: SlagFigAndelerGrVar.R}
Funksjon som genererer en figur med andeler av en variabel for en valgt grupperingsvariabel,
f.eks. sykehus.
Funksjonen er delvis skrevet for å kunne brukes til andre grupperingsvariable enn sykehus.

\begin{figure}
\centering
\includegraphics[width=0.7\linewidth]{Aarsrapp2013/InnlSlagenh}
\caption{Andel pasienter som ble innlagt direkte i slagenhet, 2013}
\label{fig:InnlSlagenh}
\end{figure}


FigAndelerGrVar <- function(RegData, valgtVar, datoFra='2013-01-01', datoTil='2050-12-31', 
minald=0, maxald=130, erMann='', diagnose='', innl4t='', NIHSSinn='', libkat, outfile='')

Figuren er definert for følgende variable:  \\
"BehSlagenhet"        "InnlInnen4eSymptom"  "InnlSlagenh"         "LipidI63u80"        
"SvelgtestUtfort"     "TrombolyseI63"       "UtAntikoagI63atrie"  "UtAntitrombotiskI63"
"UtBT" 


\subsection{Funksjon: SlagFigMeanMed.R}
Funksjon som genererer en figur med gjennomsnitt/median for hvert sykehus og kan ta inn ulike numeriske variable.
Funksjonen er delvis skrevet for å kunne brukes til andre grupperingsvariable enn sykehus.
Figurdefinisjon
FigMeanMed <- function(RegData, valgtVar, valgtMaal='Gjsn', datoFra='2012-04-01', datoTil='2050-12-31', minald=0, maxald=130, erMann='', diagnose='', innl4t='', NIHSSinn='', libkat, outfile='') 

\begin{figure}
\centering
\includegraphics[width=0.7\linewidth]{Aarsrapp2013/AlderGjsn}
\caption{Gjennomsnittsalder ved hver enhet}
\label{fig:AlderGjsn}
\end{figure}


Figuren er definert for følgende variable: \\
Alder, AntDagerInnl, NIHSSinnkomst, NIHSSpreTrombolyse, NIHSSetterTrombolyse, NIHSSpreTrombektomi, NIHSSetterTrombektomi, TidSymptInnlegg, TidSymptTrombolyse, TidInnleggTrombolyse

\section{Bibliotekfunksjoner}


\section{Samledokument}

\end{document}